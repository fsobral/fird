\documentclass[a4paper,12pt]{article}

\usepackage[utf8]{inputenc}
\usepackage{amsmath,amsfonts}
\usepackage{enumerate}
\usepackage{hyperref}

\newcommand{\fkss}{\textsc{FKSS}}
\newcommand{\R}{\mathbb{R}}

\title{\fkss\ -- User guide}
\author{}
\date{}

\begin{document}

\maketitle

\tableofcontents

\section{Introduction}

The present document contains information for users interested in
\begin{enumerate}[(i)]
\item running the algorithm~\fkss;
\item using~\fkss\ to solve his/her own problem;
\item reproducing the massive tests;
\item changing the solvers and methods used in the Feasibility and
  Optimality phases;
\item improving~\fkss.
\end{enumerate}

For the description of the algorithm, theoretical results and
numerical implementation details, please see
\begin{quote}
  ``Global convergence of a derivative-free inexact restoration filter
  algorithm for nonlinear programming'', P.S. Ferreira, E.W. Karas,
  M. Sachine and F.N.C. Sobral, Submitted, 2016.
\end{quote}

\section{Download and instalation} \label{download}

In order to work correctly, \fkss\ needs methods for Feasibility Phase
and for solving quadract programming problems. In its current version,
\fkss\ uses the Augmented Lagrangian solver Algencan~\cite{anbimasc07}
for both cases.

The steps for building \fkss\ with Algencan are described bellow.

\paragraph{Building Algencan libraries}
\begin{enumerate}
\item Download Algencan from the TANGO Web page:
  \begin{center}
    \url{http://www.ime.usp.br/~egbirgin/tango}
  \end{center}
\item Follow the instructions to build file
  \verb+$ALGSRC/lib/libalgencan.a+, where \verb+$ALGSRC+ is the full
  path to the place Algencan has been installed.
\end{enumerate}

\paragraph{Building~\fkss}
\begin{enumerate}
\item Download or clone the GitHub repository located in
  \begin{center}
    \url{https://github.com/fsobral/fkss}
  \end{center}
\item Enter in the directory \verb+fkss+ and edit file
  \verb+Makefile+, setting variable \verb+SOLVERLIB+ to the
  \underline{location} of \verb+libalgencan.a+, e.g.:
  \begin{center}
    \verb+SOLVERLIB = $ALGSRC/lib+,
  \end{center}
  where \verb+ALGSRC+ is the full path to the place Algencan has been
  installed;
\item Type
  \begin{center}
    \verb+make fkss+
  \end{center}
  An executable file \verb+fkss+ will be created in \verb+bin/+
  directory. This file is problem HS14 from the Hock-Schittkowski test
  collection~\cite{Hock1980};
\item Run the problem by typing
  \begin{center}
    \verb+bin/fkss+
  \end{center}
\end{enumerate}

Users are encouraged to modify the examples contained in directory
\verb+tests/examples/+ in order to solve their own problems. To build
a new problem, simply type
\begin{center}
  \verb+make fkss PROBLEM=full/path/to/file.f90+
\end{center}
in the root directory of \fkss. Then, type \verb+bin/fkss+ to run the
algorithm.

\section{Solving problems} \label{solving}

Algorithm~\fkss\ is capable of solving the following optimization
problem:
\begin{equation} \label{solving:prob}
  \begin{array}{lr}
    \mbox{Minimize} & f(x) \\
    \mbox{subject to} & {c_{\cal E}}(x) = 0 \\
                      & {c_{\cal I}}(x) \leq 0,
  \end{array}
\end{equation}
where the objective function $f:\R^n \rightarrow \R$ and the functions
$c_i:\R^n \rightarrow \R,$ for $i \in {\cal E} \cup {\cal I},$ that
define the constraints are continuously differentiable, but the
derivatives of $f$ are not available.

\subsection{The subroutines}

In order to solve his/her own problem, users have to provide 3
subroutines
\begin{itemize}
\item objective function;
\item constraints;
\item gradient of constraints.
\end{itemize}
Each subroutine interface is described in file
\verb+src/userinterface.f90+.

\subsubsection{The objective funtion}

The subroutine which evaluates the objective function is given by
\begin{center}
\verb+USERF(N,X,F,FLAG)+
\end{center}
where
\begin{itemize}
\item \verb+N+: (input) number of variables
\item \verb+X+: (input) \verb+N+-dimensional vector with the point in
  which the function will be evaluated
\item \verb+F+: (output) objective function value at \verb+X+
\item \verb+FLAG+: (output) should be 0 if no errors occur and any
  value otherwise
\end{itemize}

\section{Changing solvers} \label{solvers}

\bibliography{library}
\bibliographystyle{alpha}

\end{document}
